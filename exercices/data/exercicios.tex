\documentclass{article}
\usepackage{graphicx}
\usepackage[brazil]{babel}
\usepackage{siunitx}
\title{Exercícios WorkShop de Python}
\author{Jonas Tieppo da Rocha}
\date{10/09/2024}

\begin{document}

\maketitle


\begin{enumerate}
    \item O arquivo \verb|.\polimero\Specimen_RawData_1.csv| possuí dados de ensaio de um corpo de prova de PEAD. Para tal espécime:
    \begin{enumerate}
        \item Leia os dados dos arquivos, utilizando o pandas ou o \verb|ReadExperimentalData|;
        \item Visualize os dados de força e deslocamento através de alguma biblioteca (matplotlib, searborn, plotly, por exemplo);
        \item Limpe, se necessário, os dados e os transforme em termos de tensão e deformação. Para tal, considere uma espessura $t=\qty{3}{mm}$, largura $w_0=\qtylist{13}{mm}$ e comprimento $L_0=\qty{57}{mm}$.
        \item Encontre o módulo de elasticidade e a tensão de escoamento.
    \end{enumerate}
    \item Utilizando a classe \verb |SeveralMechanicalTestingFittingLinear|, sabendo-se que o tipo da máquina é \verb|machineName = `68FM100'|,
    e o material é \verb|materialType=`polymer'|, leia os dados dos corpos de prova localizados em \verb|\data\several_polymer| e analise os dados mecânicos.

    \item Utilizando a classe \verb |SeveralMechanicalTestingFittingLinear|, sabendo-se que o tipo da máquina é \verb|machineName = `68FM100_biaxial'|,
    e o material é \verb|materialType=`composite'|, leia os dados dos corpos de prova localizados em \verb|\data\several_composite| e analise os dados mecânicos. Utilize a norma ASTM-D3039. Para isso, configure 
    o \newline \verb|calculus_method=`standard-ASTM-D3039'|. Não esqueceça de colocar \verb|direction = `parallel'|

\end{enumerate}








\end{document}